\documentclass[12pt,a4paper]{article}
\usepackage[top=1cm, bottom=1.5cm, left=1cm, right=1cm]{geometry}
\usepackage[utf8]{inputenc}
\usepackage[french]{babel}
\usepackage[T1]{fontenc}
\usepackage{url}

\begin{document}

\title{Corpus et méthode expérimentale : projet}
\author{Florian Boudin}

\maketitle

\section*{Méthodologie à suivre}

\begin{enumerate}
    \item Lecture d'articles
    \begin{enumerate}
        \item Kim et al., \textit{Baldwin Semeval-2010 task 5: Automatic keyphrase extraction from scientific articles}
        \\ \url{http://www.aclweb.org/anthology/S10-1004}
        \item Kim et al., \textit{Evaluating N-gram based Evaluation Metrics for Automatic Keyphrase Extraction} 
        \\ \url{http://www.aclweb.org/anthology/C10-1065}
        \item Hasan and Ng, \textit{Automatic Keyphrase Extraction : A Survey of the State of the Art}
        \\ \url{http://www.aclweb.org/anthology/P14-1119}
        \item Bougouin et al., \textit{TopicRank: Graph-Based Topic Ranking for Keyphrase Extraction}
        \\ \url{http://www.aclweb.org/anthology/I13-1062}

        % \item Radev et Abu-Jbara, \textit{Rediscovering ACL discoveries through the lens of ACL anthology network citing sentences}, \url{http://www.aclweb.org/anthology-new/W/W12/W12-3201.pdf}.
        % \item Teufel, Siddharthan et Tidhar, \textit{Automatic classification of citation function}, \url{http://www.aclweb.org/anthology-new/W/W06/W06-1613.pdf}.
        % \item Qazvinian et Radev, \textit{Identifying Non-Explicit Citing Sentences for Citation-Based Summarization}, \url{http://www.aclweb.org/anthology-new/P/P10/P10-1057.pdf}.
        % \item Elkiss, Shen, Fader, Erkan, States et Radev. \textit{Blind men and elephants : What do citation summaries tell us about a research article?}, \url{http://homes.cs.washington.edu/~afader/bib_pdf/jasist08.pdf}
    \end{enumerate}                                                                                                               
    \item Analyse du corpus \texttt{WikinewsKeyphraseCorpus}                   
    \begin{itemize}
        \item \url{https://github.com/adrien-bougouin/WikinewsKeyphraseCorpus}
        \item Caractéristiques du corpus~: nature des documents, droits de redistribution, taille, etc.
        \item Annotation du corpus~: \textit{guidelines}, points faibles
    \end{itemize}
    \item Modification du corpus
    \begin{itemize}
        \item Proposition de nouvelles \textit{guidelines}
        \item Modification du corpus existant et ajout de nouveaux documents
    \end{itemize}
    \item Valorisation du corpus créé
    \begin{itemize}
        \item e.g.~article soumis à LREC, RECITAL
    \end{itemize}

\end{enumerate}





\end{document}  